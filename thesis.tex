%%
%% This is file `sample-sigconf.tex',
%% generated with the docstrip utility.
%%
%% The original source files were:
%%
%% samples.dtx  (with options: `all,proceedings,bibtex,sigconf')
%% 
%% IMPORTANT NOTICE:
%% 
%% For the copyright see the source file.
%% 
%% Any modified versions of this file must be renamed
%% with new filenames distinct from sample-sigconf.tex.
%% 
%% For distribution of the original source see the terms
%% for copying and modification in the file samples.dtx.
%% 
%% This generated file may be distributed as long as the
%% original source files, as listed above, are part of the
%% same distribution. (The sources need not necessarily be
%% in the same archive or directory.)
%%
%%
%% Commands for TeXCount
%TC:macro \cite [option:text,text]
%TC:macro \citep [option:text,text]
%TC:macro \citet [option:text,text]
%TC:envir table 0 1
%TC:envir table* 0 1
%TC:envir tabular [ignore] word
%TC:envir displaymath 0 word
%TC:envir math 0 word
%TC:envir comment 0 0
%%
%% The first command in your LaTeX source must be the \documentclass
%% command.
%%
%% For submission and review of your manuscript please change the
%% command to \documentclass[manuscript, screen, review]{acmart}.
%%
%% When submitting camera ready or to TAPS, please change the command
%% to \documentclass[sigconf]{acmart} or whichever template is required
%% for your publication.
%%
%%
\documentclass[sigconf]{acmart}
\usepackage{lmodern}

%% Rights management information.  This information is sent to you
%% when you complete the rights form.  These commands have SAMPLE
%% values in them; it is your responsibility as an author to replace
%% the commands and values with those provided to you when you
%% complete the rights form.
\setcopyright{rightsretained}
\copyrightyear{2025}

%%
%% end of the preamble, start of the body of the document source.
\begin{document}

%%
%% The "title" command has an optional parameter,
%% allowing the author to define a "short title" to be used in page headers.
\title{Title Goes Here}

\author{Clarity Shimoniak}
\email{clarity.shimoniak@email.ucr.edu}
\affiliation{%
	\institution{University of California, Riverside}
	\city{Riverside}
	\state{California}
	\country{USA}
}

%%
%% The abstract is a short summary of the work to be presented in the
%% article.
\begin{abstract}
% Abstract should be double-spaced and limited to 350 words or 2,450 characters.
Both FPGAs and RDMA have seen increasing adoption in datacenters as a means of overcoming the bottlenecks of parallelism needed in modern applications. We propose a novel database architecture built around a distributed FPGA system with RDMA as its interconnect in order to implement high-performance, in-memory key-value store.
\end{abstract}


%%
%% This command processes the author and affiliation and title
%% information and builds the first part of the formatted document.
\maketitle

\section{Introduction}

\subsection{FPGA}

Field-Programmable Gate Arrays

that have seen increasing adoption with decreasing costs.

One application that FPGAs are uniquely for is networking, as the dataflow programming model aligns well with the way that streams of data are processed in high throughput networking situations. FPGAs have seen increasing deployment in datacenters as a means to improve the underlying datacenter infrastructure \cite{bobda-trets-2022}, rather than simply as accelerators as GPUs are.


\subsection{RDMA}

Remote Direct Memory Access (RDMA) is an extension to the concept of direct memory access (DMA), which a system's memory to be accessed without the involvement of its CPU. RDMA is a standard allowing for such transactions to take place over a network rather than a local like like PCIe. Compared to traditional networking protocols, RDMA is significantly faster, moving the bottleneck of distributed systems out of the networking portion and into processing portion \cite{binnig-vldb-2016}.

RDMA has also seen widespread datacenter adoption, though primarily on CPU-based systems that use specialized network interface cards (NICs) to handle RDMA operations.

\citet{star} has shown the viability of FPGAs as network accelerators, but \citeauthor{star} use them to implement custom a NIC rather than as part of an application \cite{star}.

RDMA operations can be either one-sided or two-sided. One-sided operations access memory at a specific location on the remote node. These are lightweight and simple to implement, but are more difficult for applications to use. Two-sided operations  \cite{base}.

For CPU systems, there is a tradeoff between the two types of operations. Using an FPGA


\subsection{B-Link Tree}

B-Link trees are an extension to B+ trees proposed by \citeauthor{b-link} to support concurrency. Like B+ trees, they are self-balancing data structures with an adjustable fan-out factor that store all data at leaf nodes. B-Link trees introduce additional linkages between nodes and ensure that no more than three nodes are locked at a time-per transaction \cite{b-link}.

\input{chapter2}
\section{Chapter 3 Title}

\section{Chapter 4 Title}




%%
%% The acknowledgments section is defined using the "acks" environment
%% (and NOT an unnumbered section). This ensures the proper
%% identification of the section in the article metadata, and the
%% consistent spelling of the heading.
\begin{acks}
Acknowledgments go here
\end{acks}


%%
%% The next two lines define the bibliography style to be used, and
%% the bibliography file.
\bibliographystyle{ACM-Reference-Format}
\bibliography{bibfile}


%%
%% If your work has an appendix, this is the place to put it.
\appendix

\sect{Design Utilization}

\begin{table}[H]
	\centering
	\begin{tabular}{c|ccccccccc}
		& \textbf{LUT} & \textbf{Register} & \textbf{BRAM} & \textbf{URAM} & \textbf{DSP} \\
		\hline
		Count & 38,827 & 38,729 & 24 & 0 & 5 \\
		Percentage & 2.98 & 1.71 & 1.19 & 0.0 & 0.06 \\
	\end{tabular}
	\caption{Design Resource Usage}
	\label{table:resource-usage}
\end{table}

\begin{figure}[H]
	\includegraphics[width=0.9\textwidth]{floorplan}
	\caption{Design Floorplan}
	\label{fig:floorplan}
\end{figure}



\end{document}
\endinput
%%
%% End of file `sample-sigconf.tex'.
