\sect{Experiment}

FPGA data was collected on a system with a Xilinx Alveo U280 PCIe card,
containing a XCU280 FPGA and $\SI{8}{\giga\byte}$ of on-chip high-bandwidth
memory (HBM) \autocite{u280}. The CPU baseline was collected on a system with an
Intel Core i9-9900 CPU and $\SI{32}{\giga\byte}$ of installed RAM running Ubuntu
20.04 LTS.


\subsect{Threads vs. Modules}

Parallel processing on CPUs uses threads as a unit of execution. There is no
hard limit to how many threads that a program can create, but there will be no
speedup from using more threads than the physical CPU can support. Additionally,
threads sharing physical cores through simultaneous multithreading will not
perform as well as threads running on their own hardware cores.

On an FPGA, all processing is inherently parallel unless there is a data
dependency between one operation and another. The ceiling to FPGA parallelism is
the number of instances of a module that can fit into the FPGA's fabric.

Using HLS to run identical code and comparing it to a traditional multi-threaded
version shows \todo{\ldots}


\subsect{Traditional vs. Optimistic Locking}

As discussed in \autoref{subsec:concurrency}, modern CPU designs often avoid
traditional locks types because of the heavy penalty of unnecessary cache
invalidations. In \citeyear{b-link} when \citeauthor{b-link} designed the B-Link
tree, multi-core machines were still decades away, but in the modern era even
many microcontrollers have multiple cores.

However, FPGAs do not have a native caching protocol and will perform writes
directly unless a specific caching protocol is implemented by the designer.
Thus, the penalty for traditional locking is lower. Traditional locking has the
benefit that it avoids an unbounded number of re-attempts at performing a write
in a write-heavy workload.


\subsect{Read-Dominated vs. Write-Dominated}


\subsect{Node Allocation Strategies}
