\sect{Experiment}

\subsect{Setup}

FPGA data was collected on a system with a Xilinx Alveo U280 PCIe card,
which contains an XCU280 FPGA with $\SI{8}{\giga\byte}$ of on-chip HBM as well
as $\SI{32}{\giga\byte}$ DDR4 memory on two RDIMMs operating at
$\SI{2400}{\mega{T}\per\second}$ \autocite{u280}. The CPU baseline was collected
on a system with an AMD Ryzen 7 7700X CPU and $\SI{96}{\giga\byte}$ of DDR5
memory on four UDIMMs operating at $\SI{3600}{\mega{T}\per\second}$. The system
is running AlmaLinux 9.6 (binary compatible with Red Hat Enterprise Linux) and
kernel version 5.14.0.
%
Unless stated otherwise, a branching factor of $m=8$ will be used in all
experiments.


\subsect{Tree Fanout}


\begin{figure}[H]
	\centering
	\begin{subfigure}{7.5cm}
		\centering
		\begin{tikzpicture}[scale=0.8]
	\begin{axis}
		[
			xlabel={Input Size (Requests)},
			ylabel={Speedup},
			legend style={
				at={(0.5, 1.05)},
				anchor=south,
				legend columns=3
			},
			domain=1:1000000,
			xmode=log,
			ymode=log,
			ymin=0.2, ymax=300,
			cycle list name=rainbow,
		]
		\addplot coordinates {
			(1000, 0.447992100065833)
			(10000, 2.42801339285714)
			(100000, 21.0124932102118)
			(1000000, 164.431034482759)
		};
		\addlegendentry{$m=4$}
		\addplot coordinates {
			(1000, 0.370357142857143)
			(10000, 2.56777996070727)
			(100000, 18.0950704225352)
			(1000000, 140.34672489083)
		};
		\addlegendentry{$m=8$}
		\addplot coordinates {
			(1000, 0.986560992419021)
			(10000, 6.13464741810105)
			(100000, 55.1531266702298)
			(1000000, 112.808505747126)
		};
		\addlegendentry{$m=16$}
		\addplot coordinates {
			(1000, 1.06377730796335)
			(10000, 7.32187314759929)
			(100000, 62.1744820065431)
			(1000000, 106.099778761062)
		};
		\addlegendentry{$m=32$}
	\end{axis}
\end{tikzpicture}

		\caption{Sequential Insertion}
		\label{fig:fpga-sequential-insert}
	\end{subfigure}
	\begin{subfigure}{7.5cm}
		\centering
		\begin{tikzpicture}[scale=0.8]
	\begin{axis}
		[
			xlabel={Input Size (Requests)},
			ylabel={Speedup},
			legend style={
				at={(0.5, 1.05)},
				anchor=south,
				legend columns=3
			},
			domain=1:1000000,
			xmode=log,
			ymode=log,
			ymin=0.2, ymax=300,
			cycle list name=rainbow,
		]
		\addplot coordinates {
			(1000, 0.309157959434542)
			(10000, 2.1794101583834)
			(100000, 23.620082815735)
			(1000000, 214.608667247387)
		};
		\addlegendentry{$m=4$}
		\addplot coordinates {
			(1000, 0.4024)
			(10000, 2.47738835500283)
			(100000, 20.7277472527473)
			(1000000, 164.921340629275)
		};
		\addlegendentry{$m=8$}
		\addplot coordinates {
			(1000, 1.39441591784339)
			(10000, 9.47141344626787)
			(100000, 94.0264305177112)
			(1000000, 122.260208926876)
		};
		\addlegendentry{$m=16$}
		\addplot coordinates {
			(1000, 1.86494638069705)
			(10000, 12.2645714285714)
			(100000, 108.964765100671)
			(1000000, 71.4358802502234)
		};
		\addlegendentry{$m=32$}
	\end{axis}
\end{tikzpicture}

		\caption{Random Insertion}
		\label{fig:fpga-random-insert}
	\end{subfigure}
	\caption{Insertion Speedup}
	\label{fig:insert-speedup}
\end{figure}

\begin{figure}[H]
	\centering
	\begin{subfigure}{7.5cm}
		\centering
		\begin{tikzpicture}[scale=0.8]
	\begin{axis}
		[
			xlabel={Input Size (Requests)},
			ylabel={Speedup},
			legend style={
				at={(0.5, 1.05)},
				anchor=south,
				legend columns=3
			},
			domain=1:1000000,
			xmode=log,
			ymode=log,
			cycle list name=rainbow,
		]
		\addplot coordinates {
			(1000, 3.16878358573522)
			(10000, 77.4286605657238)
			(100000, 89.4915845648604)
			(1000000, 179.639907499174)
		};
		\addlegendentry{$m=4$}
		\addplot coordinates {
			(1000, 1.10674964097654)
			(10000, 2.2621107266436)
			(100000, 14.0972812234495)
			(1000000, 103.548872180451)
		};
		\addlegendentry{$m=8$}
		\addplot coordinates {
			(1000, 1.70931349015843)
			(10000, 19.9047835990888)
			(100000, 1303.46703955254)
			(1000000, 89327.134214876)
		};
		\addlegendentry{$m=16$}
		\addplot coordinates {
			(1000, 1.94208686969546)
			(10000, 13.3864785172704)
			(100000, 64.7089068825911)
			(1000000, 5952.13113948919)
		};
		\addlegendentry{$m=32$}
	\end{axis}
\end{tikzpicture}

		\caption{Sequential Search}
		\label{fig:fpga-sequential-search}
	\end{subfigure}
	\begin{subfigure}{7.5cm}
		\centering
		\begin{tikzpicture}[scale=0.8]
	\begin{axis}
		[
			xlabel={Input Size (Requests)},
			ylabel={Speedup},
			legend style={
				at={(0.5, 1.05)},
				anchor=south,
				legend columns=3
			},
			domain=1:1000000,
			xmode=log,
			ymode=log,
			cycle list name=rainbow,
		]
		\addplot coordinates {
			(1000, 3.10217176702863)
			(10000, 151.591523341523)
			(100000, 14608.754)
			(1000000, 209854.983388704)
		};
		\addlegendentry{$m=4$}
		\addplot coordinates {
			(1000, 2.83683048039757)
			(10000, 68.3478442863123)
			(100000, 5096.78003280033)
			(1000000, 601188.418580376)
		};
		\addlegendentry{$m=8$}
		\addplot coordinates {
			(1000, 18.9140546006067)
			(10000, 1325.56059356966)
			(100000, 111334.508031301)
			(1000000, 1958.50923128567)
		};
		\addlegendentry{$m=16$}
		\addplot coordinates {
			(1000, 13.8760069044879)
			(10000, 64.5318648417451)
			(100000, 9943.00341914722)
			(1000000, 200399.102272727)
		};
		\addlegendentry{$m=32$}
	\end{axis}
\end{tikzpicture}

		\caption{Random Search}
		\label{fig:fpga-random-search}
	\end{subfigure}
	\caption{Search Speedup}
	\label{fig:search-speedup}
\end{figure}


\subsect{Workload Balance}

\begin{figure}[H]
	\centering
	\begin{tikzpicture}
	\begin{axis}
		[
			xlabel={Request Count},
			ylabel={Execution Time ($\unit{\milli\second}$)},
			legend pos=south east,
			domain=1:1000000,
			xmode=log,
			ymode=log,
		]
		\addplot+[cycle list name=color list]
			coordinates {
				(1000, 2.626)
				(10000, 206.79625)
				(100000, 11410.7)
				(1000000, 117235.5)
			};
		\addlegendentry{$\SI{20}{\percent}$ Read, $\SI{80}{\percent}$ Write}
		\addplot+[cycle list name=color list]
			coordinates {
				(1000, 1.731)
				(10000, 140.31525)
				(100000, 8097.0275)
				(1000000, 91322.3)
			};
		\addlegendentry{$\SI{40}{\percent}$ Read, $\SI{60}{\percent}$ Write}
		\addplot+[cycle list name=color list]
			coordinates {
				(1000, 1.20775)
				(10000, 89.748)
				(100000, 4969.515)
				(1000000, 52228.5)
			};
		\addlegendentry{$\SI{60}{\percent}$ Read, $\SI{40}{\percent}$ Write}
		\addplot+[cycle list name=color list]
			coordinates {
				(1000, 0.76375)
				(10000, 44.85)
				(100000, 2507.505)
				(1000000, 28785.65)
			};
		\addlegendentry{$\SI{80}{\percent}$ Read, $\SI{20}{\percent}$ Write}
	\end{axis}
\end{tikzpicture}

	\caption{Read-Write Balance for a CPU Tree with $m=4$}
	\label{fig:rw-balance}
\end{figure}


\subsect{HBM Speedup}

One unique benefit of using an accelerator card rather than a CPU is the ability
to use High-Bandwidth Memory (HBM) for main system memory rather than being
limited to the Double-Data Rate (DDR) that is used on essentially all CPU
systems. However, the U280 has both technologies available, so the effect of
this variable can be isolated from the computation itself. When allocating
memory resources, in both cases the tree's data memory was given half of the
bandwidth, while the request and response buffers were each given a quarter of
the bandwidth.

\begin{figure}[H]
	\centering
	\begin{subfigure}{7.5cm}
		\centering
		\begin{tikzpicture}[scale=0.8]
	\begin{axis}
		[
			xlabel={Input Size (Requests)},,
			ylabel={HBM Speedup over DDR},
			legend style={
				at={(0.5, 1.05)},
				anchor=south,
				legend columns=4
			},
			domain=1:1000000,
			xmode=log,
			ymin=1, ymax=2.4,
			ytick distance=0.2,
			cycle list name=rainbow,
		]
		\addplot coordinates {
			(1000, 1.62113127001067)
			(10000, 1.77250247279921)
			(100000, 1.90382626680455)
			(1000000, 1.89852700490998)
		};
		\addlegendentry{$m=4$}
		\addplot coordinates {
			(1000, 1.50537634408602)
			(10000, 1.67250821467689)
			(100000, 1.74645222327342)
			(1000000, 2.30150753768844)
		};
		\addlegendentry{$m=8$}
		\addplot coordinates {
			(1000, 1.55021367521368)
			(10000, 1.8028028028028)
			(100000, 1.72920517560074)
			(1000000, 1.80049668874172)
		};
		\addlegendentry{$m=16$}
		\addplot coordinates {
			(1000, 1.61801596351197)
			(10000, 1.66535044422507)
			(100000, 1.9489904357067)
			(1000000, 1.94158075601375)
		};
		\addlegendentry{$m=32$}
	\end{axis}
\end{tikzpicture}

		\caption{Sequential Insertion}
		\label{fig:hbm-speedup-sequential-insertion}
	\end{subfigure}
	\begin{subfigure}{7.5cm}
		\centering
		\begin{tikzpicture}[scale=0.8]
	\begin{axis}
		[
			xlabel={Input Size (Requests)},
			ylabel={DDR Speedup over HBM},
			legend style={
				at={(0.5, 1.05)},
				anchor=south,
				legend columns=4
			},
			domain=1:1000000,
			xmode=log,
			ymin=1, ymax=2.4,
			ytick distance=0.2,
			cycle list name=rainbow,
		]
		\addplot coordinates {
			(1000, 1.78791208791209)
			(10000, 1.93756613756614)
			(100000, 1.98970133882595)
			(1000000, 1.80787401574803)
		};
		\addlegendentry{$m=4$}
		\addplot coordinates {
			(1000, 1.50602409638554)
			(10000, 1.75322101090188)
			(100000, 1.98907103825137)
			(1000000, 1.8616298811545)
		};
		\addlegendentry{$m=8$}
		\addplot coordinates {
			(1000, 1.71775082690187)
			(10000, 1.85925196850394)
			(100000, 1.73768939393939)
			(1000000, 1.80772532188841)
		};
		\addlegendentry{$m=16$}
		\addplot coordinates {
			(1000, 1.64498346196251)
			(10000, 1.8210197710718)
			(100000, 2.00448430493274)
			(1000000, 1.22228290551611)
		};
		\addlegendentry{$m=32$}
	\end{axis}
\end{tikzpicture}

		\caption{Random Insertion}
		\label{fig:hbm-speedup-random-insertion}
	\end{subfigure}
	\caption{DDR/HBM Insertion Speedup}
	\label{fig:hbm-insertion-speedup}
\end{figure}

\begin{figure}[H]
	\centering
	\begin{subfigure}{7.5cm}
		\centering
		\begin{tikzpicture}[scale=0.8]
	\begin{axis}
		[
			xlabel={Input Size (Requests)},
			ylabel={DDR Speedup over HBM},
				legend style={
				at={(0.5, 1.05)},
				anchor=south,
				legend columns=4
			},
			domain=1:1000000,
			xmode=log,
			ymin=0.9, ymax=2.4,
			ytick distance=0.2,
			cycle list name=rainbow,
		]
		\addplot coordinates {
			(1000, 1.16306818181818)
			(10000, 1.38638985005767)
			(100000, 1.4692400482509)
			(1000000, 1.55950540958269)
		};
		\addlegendentry{$m=4$}
		\addplot coordinates {
			(1000, 1.27924066135946)
			(10000, 1.34028985507246)
			(100000, 1.46211180124224)
			(1000000, 1.61925843940232)
		};
		\addlegendentry{$m=8$}
		\addplot coordinates {
			(1000, 1.32675159235669)
			(10000, 1.28137769994162)
			(100000, 1.51330108827086)
			(1000000, 1.58876050420168)
		};
		\addlegendentry{$m=16$}
		\addplot coordinates {
			(1000, 1.22432762836186)
			(10000, 1.27292225201072)
			(100000, 1.48259303721489)
			(1000000, 1.59811616954474)
		};
		\addlegendentry{$m=32$}
	\end{axis}
\end{tikzpicture}

		\caption{Sequential Search}
		\label{fig:hbm-speedup-sequential-search}
	\end{subfigure}
	\begin{subfigure}{7.5cm}
		\centering
		\begin{tikzpicture}[scale=0.8]
	\begin{axis}
		[
			xlabel={Input Size (Requests)},
			ylabel={DDR Speedup over HBM},
			legend style={
				at={(0.5, 1.05)},
				anchor=south,
				legend columns=4
			},
			domain=1:1000000,
			xmode=log,
			ymin=0.9, ymax=2.4,
			ytick distance=0.2,
			cycle list name=rainbow,
		]
		\addplot coordinates {
			(1000, 1.27021943573668)
			(10000, 1.44326241134752)
			(100000, 1.62654521795706)
			(1000000, 1.58421052631579)
		};
		\addlegendentry{$m=4$}
		\addplot coordinates {
			(1000, 1.21952861952862)
			(10000, 1.53042921204356)
			(100000, 1.72856130403969)
			(1000000, 1.66319444444444)
		};
		\addlegendentry{$m=8$}
		\addplot coordinates {
			(1000, 1.16972205795387)
			(10000, 1.32931506849315)
			(100000, 1.35566722501396)
			(1000000, 1.53398558187436)
		};
		\addlegendentry{$m=16$}
		\addplot coordinates {
			(1000, 0.975856260527793)
			(10000, 1.40758579169175)
			(100000, 1.60594315245478)
			(1000000, 1.56297093649085)
		};
		\addlegendentry{$m=32$}
	\end{axis}
\end{tikzpicture}

		\caption{Random Search}
		\label{fig:hbm-speedup-random-search}
	\end{subfigure}
	\caption{DDR/HBM Search Speedup}
	\label{fig:hbm-search-speedup}
\end{figure}

As shown in \autoref{fig:hbm-insertion-speedup} \&
\autoref{fig:hbm-search-speedup}, in nearly all cases there is a performance
benefit that can in the best cases result in a doubling of performance.
Insertions benefit more than searches due to better parallelism of read and
write accesses.


\subsect{Threads vs. Modules}

Parallel processing on CPUs uses threads as a unit of execution. There is no
hard limit to how many threads that a program can create, but there will be no
speedup from using more threads than the physical CPU can support. Additionally,
threads sharing physical cores through simultaneous multithreading will not
perform as well as threads running on their own hardware cores.

On an FPGA, all processing is inherently parallel unless there is a data
dependency between one operation and another. The ceiling to FPGA parallelism is
the number of instances of a module that can fit into the FPGA's fabric.

Using HLS to run identical code and comparing it to a traditional multi-threaded
version shows \todo{\ldots}


\subsect{Traditional vs. Optimistic Locking}

As discussed in \autoref{subsec:concurrency}, modern CPU designs often avoid
traditional lock types because of the heavy penalty of unnecessary cache
invalidations. In \citeyear{b-link} when \citeauthor{b-link} designed the B-Link
tree, multi-core machines were still decades away, but in the modern era even
many microcontrollers have multiple cores.

However, FPGAs do not have a native caching protocol and will perform writes
directly unless a specific caching protocol is implemented by the designer.
Thus, the penalty for traditional locking is lower. Traditional locking has the
benefit that it avoids an unbounded number of re-attempts at performing a write
in a write-heavy workload.


\subsect{Read-Dominated vs. Write-Dominated}


\subsect{Node Allocation Strategies}
