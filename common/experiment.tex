\sect{Experiment}

\subsect{Setup}

FPGA data was collected on a system with a Xilinx Alveo U280 PCIe card,
which contains an XCU280 FPGA with $\SI{8}{\giga\byte}$ of on-chip HBM as well
as $\SI{32}{\giga\byte}$ DDR4 memory on two RDIMMs operating at
$\SI{2400}{\mega{T}\per\second}$ \autocite{u280}. The CPU baseline was collected
on a system with an Intel Core i9-9900 CPU and $\SI{32}{\giga\byte}$ of DDR4.
The system is running Ubuntu 20.04 LTS.

The C code used to synthesize the tree under HLS is the same code used to obtain
the CPU baseline. The C version uses pthreads for locking and concurrency
support. Because a pthread mutex takes more space than a lock bit, the CPU
version uses more memory per node. Though running as a single-threaded program,
locks are still acquired and released in order to demonstrate their negative
impact on caching.
%
Unless stated otherwise, a branching factor of $m=4$ will be used in all
experiments.


\subsect{Methodology}

In both CPU and FPGA cases, computation time measures only the time that the
database engine spends processing data, not the time spent preparing the
request, response, and memory buffers. Timing data is collected by the testbench
program with C++'s \texttt{std::chrono} library.
%
Request data was stored in pre-generated binary files that are loaded by a
testbench program, which also initializes the main memory buffer that will store
the tree.
% 
Before running a search or mixed request file, an insert request file of the
same size is run to populate the tree.
%
In both cases, only a single ``processor'' was dedicated to the tree engine: on
the CPU this means running on a single thread and on the FPGA this means running
with one kernel instance.


\subsect{Comparison}

\begin{figure}[H]
	\centering
	\begin{subfigure}{7.5cm}
		\centering
		\begin{tikzpicture}[scale=0.8]
	\begin{axis}
		[
			xlabel={Input Size (Requests)},
			ylabel={Throughput (Requests per Second)},
			legend style={
				at={(0.5, 1.05)},
				anchor=south,
				legend columns=2
			},
			domain=1:1000000,
			xmode=log,
			ymode=log,
			cycle list name=rainbow2,
		]
		\addplot coordinates {
			(1000, 4827778.20055007)
			(10000, 5381869.19452038)
			(100000, 4065170.01316331)
			(1000000, 3657002.23888243)
		};
		\addlegendentry{CPU}
		\addplot coordinates {
			(1000, 26717.4913090758)
			(10000, 257112.873988185)
			(100000, 1935652.66519638)
			(1000000, 5566292.93810852)
		};
		\addlegendentry{FPGA}
	\end{axis}
\end{tikzpicture}

		\caption{Search Throughput}
		\label{fig:search-throughput-comparison}
	\end{subfigure}
	\begin{subfigure}{7.5cm}
		\centering
		\begin{tikzpicture}[scale=0.8]
	\begin{axis}
		[
			xlabel={Input Size (Requests)},
			ylabel={Throughput (Requests per Second)},
			legend style={
				at={(0.5, 1.05)},
				anchor=south,
				legend columns=2
			},
			domain=1:1000000,
			xmode=log,
			ymode=log,
			cycle list name=rainbow2,
		]
		\addplot coordinates {
			(1000, 645287.868675484)
			(10000, 89827.589867941)
			(1000000, 3635977.12610953)
		};
		\addlegendentry{CPU}
		\addplot coordinates {
			(1000, 26640.9914354617)
			(10000, 256062.28712838)
			(100000, 1933132.34198982)
			(1000000, 5566278.98799415)
		};
		\addlegendentry{FPGA}
	\end{axis}
\end{tikzpicture}

		\caption{Insertion Throughput}
		\label{fig:insert-throughput-comparison}
	\end{subfigure}
	\caption{Throughput Comparison}
	\label{fig:throughput-comparison}
\end{figure}

As shown, the FPGA offers similar throughput for both read and write operations,
which leads to the most compelling case being write-heavy workloads. This may be
due in part to caching effects and the lack thereof on the FPGA, which at
present does not implement any caching protocol.

Testing with larger input sizes than those shown here lead to plateaus in
execution time as attempts to allocate new nodes would quickly fail. Increasing
memory size further was not possible at this time due to issues with routing and
memory allocation on the FPGA and CPU, respectively.

\begin{figure}[H]
	\centering
	\begin{subfigure}{7.5cm}
		\centering
		\begin{tikzpicture}[scale=0.8]
	\begin{axis}
		[
			xlabel={Input Size (Requests)},
			ylabel={Execution Time (Milliseconds)},
			legend style={
				at={(0.5, 1.05)},
				anchor=south,
				legend columns=2
			},
			domain=1:1000000,
			xmode=log,
			cycle list name=rainbow2,
		]
		\addplot coordinates {
			(1000, 0.0015501125)
			(10000, 0.11132575)
			(1000000, 0.275039)
		};
		\addlegendentry{CPU}
		\addplot coordinates {
			(1000, 0.03742867825)
			(10000, 0.03889342625)
			(100000, 0.0516621655)
			(1000000, 0.1796527865)
		};
		\addlegendentry{FPGA}
	\end{axis}
\end{tikzpicture}

		\caption{Search Execution Time}
		\label{fig:search-time-comparison}
	\end{subfigure}
	\begin{subfigure}{7.5cm}
		\centering
		\begin{tikzpicture}[scale=0.8]
	\begin{axis}
		[
			xlabel={Input Size (Requests)},
			ylabel={Execution Time (Seconds)},
			legend style={
				at={(0.5, 1.05)},
				anchor=south,
				legend columns=2
			},
			domain=1:1000000,
			xmode=log,
			cycle list name=rainbow2,
		]
		\addplot coordinates {
			(1000, 0.0015501125)
			(10000, 0.11132575)
			% (100000, 8.84483)
			(1000000, 0.275039)
			% (10000000, 0.15411725)
			% (100000000, 0.154009)
			% (1000000000, 0.1540275)
		};
		\addlegendentry{CPU}
		% \addplot coordinates {
		% 	(1000, 0.00020718125)
		% 	(10000, 0.001858595)
		% 	(100000, 0.024599725)
		% 	(1000000, 0.273452)
		% 	% (10000000, 0.0445418)
		% 	% (100000000, 0.044372225)
		% 	% (1000000000, 0.0443988)
		% };
		% \addlegendentry{CPU, Sequential}
		\addplot coordinates {
			(1000, 0.037536182)
			(10000, 0.0390530155)
			(100000, 0.0517295515)
			(1000000, 0.179653244)
			% (10000000, 0.32175294225)
			% (100000000, 0.32172119075)
			% (1000000000, 0.321755227)
		};
		\addlegendentry{FPGA}
		% \addplot coordinates {
		% 	(1000, 0.0376466185)
		% 	(10000, 0.03901405)
		% 	(100000, 0.05178806175)
		% 	(1000000, 0.17968360775)
		% 	% (10000000, 0.32170977325)
		% 	% (100000000, 0.32167101)
		% 	% (1000000000, 0.3217138435)
		% };
		% \addlegendentry{FPGA, Sequential}
	\end{axis}
\end{tikzpicture}

		\caption{Insertion Throughput}
		\label{fig:insert-time-comparison}
	\end{subfigure}
	\caption{Execution Time Comparison}
	\label{fig:time-comparison}
\end{figure}

Examining execution time shows that the FPGA only begins to become saturated at
under a load of $10^6$ requests, whereas the single-threaded CPU instance
immediately shows strain under load.

\begin{figure}[H]
	\centering
	\begin{subfigure}{7.5cm}
		\centering
		\begin{tikzpicture}[scale=0.8]
	\begin{axis}
		[
			xlabel={Input Size (Requests)},
			ylabel={Throughput (Requests per Second)},
			legend style={
				at={(0.5, 1.05)},
				anchor=south,
				legend columns=2
			},
			domain=1:1000000,
			xmode=log,
			ymode=log,
			cycle list name=rainbow4,
		]
		\addplot coordinates {
			(1000, 1435095.92608227)
			(10000, 197014.312573879)
			(100000, 35759.2470924166)
		};
		\addlegendentry{80/20 Read/Write}
		\addplot coordinates {
			(1000, 789949.295603965)
			(10000, 98412.2684166593)
			(100000, 18048.596384131)
		};
		\addlegendentry{60/40 Read/Write}
		\addplot coordinates {
			(1000, 511036.317321055)
			(10000, 63008.8092393335)
			(100000, 11075.1167728965)
		};
		\addlegendentry{40/60 Read/Write}
		\addplot coordinates {
			(1000, 363894.315042579)
			(10000, 43150.1129501057)
			(100000, 7877.05503667398)
		};
		\addlegendentry{20/80 Read/Write}
	\end{axis}
\end{tikzpicture}

		\caption{CPU Throughput}
		\label{fig:mixed-throughput-cpu}
	\end{subfigure}
	\begin{subfigure}{7.5cm}
		\centering
		\begin{tikzpicture}[scale=0.8]
	\begin{axis}
		[
			xlabel={Input Size (Requests)},
			ylabel={Throughput (Requests per Second)},
			legend style={
				at={(0.5, 1.05)},
				anchor=south,
				legend columns=2
			},
			domain=1:1000000,
			xmode=log,
			ymode=log,
			cycle list name=rainbow4,
		]
		\addplot coordinates {
			(1000, 26634.9166308917)
			(10000, 256720.378046995)
			(100000, 1930947.02293897)
		};
		\addlegendentry{80/20 Read/Write}
		\addplot coordinates {
			(1000, 26667.0081657027)
			(10000, 256446.31259281)
			(100000, 1933143.21462495)
		};
		\addlegendentry{60/40 Read/Write}
		\addplot coordinates {
			(1000, 26653.5229525202)
			(10000, 257318.882650967)
			(100000, 1935484.10048387)
		};
		\addlegendentry{40/60 Read/Write}
		\addplot coordinates {
			(1000, 26657.9401076127)
			(10000, 256798.857049262)
			(100000, 1934544.20568633)
		};
		\addlegendentry{20/80 Read/Write}
	\end{axis}
\end{tikzpicture}

		\caption{FPGA Throughput}
		\label{fig:mixed-throughput-fpga}
	\end{subfigure}
	\caption{Throughput Comparison in Mixed Workloads}
	\label{fig:rw-comparison}
\end{figure}
