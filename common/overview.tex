\sect{Overview}

\subsect{Database Indexing}
\label{sec:indexing}

It is common for modern databases to use a variant of the B-Tree for indexing
\autocite{ma-tpds-2022}. The self-balancing nature of B-Trees helps to control
the height of the tree. Because lookup time is proportional to the length of the
path from the root node where the search begins to the node holding the desired
data, shorter trees mean faster lookups. B+ trees are a variant of B trees the
keep all data values at the leaves rather than storing them alongside internal
nodes, which improves the average case time of tree traversal in term of number
of I/O operations.

B-Link trees are an extension to B+ trees proposed by \citeauthor{b-link} in
\citeyear{b-link} which add fields to provide for thread-safe access. Like B+
trees, they are self-balancing, have an adjustable fan-out factor, and store all
data at leaf nodes. B-Link trees introduce linkages between sibling nodes at all
levels of the tree. Principally, this is to ensure that newly created sibling
nodes are still accessible during split operations, even if they have not yet
been assigned to a parent node. This structure ensures that no more than three
nodes are locked at a time for each modification operation. However, this also
brings the benefit that range-based queries can be executed very easily, as
subsequent leaf nodes form a linked list \autocite{b-link}.


\subsect{FPGAs}
\label{sec:fpga}

Field-Programmable Gate Arrays (FPGAs) are an alternative processing element to
CPUs. Structurally, FPGAs are large arrays of reconfigurable logic gates that
can be used to implement complex digital circuits. This lower-level approach
allows for designs to exploit more parallelism than CPUs. FPGAs will not perform
as well as application-specific integrated circuits (ASICs) for an identical
design, but on-demand reconfigurability, shorter lead times, and lower cost (for
all but the largest order sizes) offsets these downsides in many applications.

A key difference of FPGAs from CPUs is how computational resources are
multiplexed. On CPU-based systems, computational resources are multiplexed
temporally; there are a limited number of instructions that can be executed each
second. FPGAs are multiplexed spatially; all processes can run simultaneously so
long as there are sufficient gates available to implement them.

Traditionally, FPGA designs are written in hardware description languages
(HDLs), with the two most prominent being Verilog and VHDL. However, in recent
years attempts have been made to change this. High-Level Synthesis (HLS)
frameworks convert programs written in traditional programming languages like C
\& C++ into HDL. This allows for easier conversion of existing algorithms and
codebases as well as increasing the pool of potential developers
\autocite{martin-destest-2009}. More recently, projects like Chisel seek to blur
the line by modifying an existing programming language, Scala, to support HDL
semantics \autocite{chisel}.

Originally, FPGAs were intended for prototyping ASICs, but later found wide
usage in the telecom industry \autocite{bobda-trets-2022,mencer-queue-2020} and
more recently in datacenters \autocite{mencer-queue-2020,hoozemans-cas-2021}.
FPGAs are uniquely suited to networking applications because the dataflow
programming model \autocite{hoozemans-cas-2021} inherent to their architecture
aligns naturally with the flow of packets in high-throughput networking
environments.


\subsect{FPGAs in the Datacenter}
\label{sec:datacenter-fpga}

FPGAs implementations of many core database operations have been developed.
Converting algorithms from CPU to FPGA is fairly straightforward
\autocite{fang-vldb-2020}, especially with the help of HLS, but optimizing them
is a focus of ongoing research \autocite{fang-vldb-2020}. Converting simple
operations like filter and projection (removing some columns from a table before
returning it) are largely a solved problem \autocite{fang-vldb-2020}, but more
complicated operations like merge and sort have been implemented in many ways,
with some showing significant performance improvements over CPU versions
\autocite{leggett-trets-2025,moghaddamfar-damon-2021}.

Much of the recent research FPGAs for database acceleration focuses on
specialized Network Interface Cards (NICs) which contain a small FPGA, called
SmartNICs. These allow some processing to be offloaded from the CPU at the point
of access to the network, but the smaller program space and reduced performance
of these FPGAs limit how much work can be moved onto them from other parts of
the system. For instance, \citetitle{honeycomb} chooses to only accelerate read
operations, assuming a read-heavy workload \autocite{honeycomb}. The structure
of FPGAs incentivises prioritizing common, simple operations because of the
resource multiplexing issues mentioned above
\autocite{honeycomb,moghaddamfar-damon-2021}.
