\sect{Related Works}

\subsect{In Software}

\citeauthor{dlsm} propose using Log-Structured Merge Trees as a
higher-performance alternative to B-trees. However, their system architecture
assumes a heterogeneous cluster with some nodes dedicated to memory and other
dedicated to processing \autocite{dlsm}.


\subsect{Hardware Acceleration}

\citeauthor{star} have shown the viability of FPGAs as network accelerators, but
use them to implement custom a NIC protocol rather than as part of an
application \autocite{star}.

Utilizing more of the U280's memory hierarchy that could be leveraged for some
of these caching effects. In addition to the $\SI{8}{\giga\byte}$ of HBM used in
this experiment, it has $\SI{32}{\giga\byte}$ of slower, off-chip DDR and
$\SI{41}{\kilo\byte}$ of faster, on-chip block RAM (BRAM) and UltraRAM (URAM).

\citetitle{honeycomb} uses a hybrid approach, with the CPU handling modification
operations and an FPGA-based smart NIC handling lookup operations. The main
reason that \citeauthor{honeycomb} give for this approach is that the cost of
accelerating modification operations was too high due to their complexity to be
beneficial in read-dominated workloads \autocite{honeycomb}. The U280 that we
are using is more capable than Honeycomb's embedded FPGA, so ``program space''
is not as much of a concern. Using Vitis HLS allows for these more complex
procedures to be described in a traditional software context, lowering the
amount of effort required to implement them effectively.
