%%
%% This is file `sample-sigconf.tex',
%% generated with the docstrip utility.
%%
%% The original source files were:
%%
%% samples.dtx  (with options: `all,proceedings,bibtex,sigconf')
%%
%% IMPORTANT NOTICE:
%%
%% For the copyright see the source file.
%%
%% Any modified versions of this file must be renamed
%% with new filenames distinct from sample-sigconf.tex.
%%
%% For distribution of the original source see the terms
%% for copying and modification in the file samples.dtx.
%%
%% This generated file may be distributed as long as the
%% original source files, as listed above, are part of the
%% same distribution. (The sources need not necessarily be
%% in the same archive or directory.)
%%
%%
%% Commands for TeXCount
%TC:macro \cite [option:text,text]
%TC:macro \citep [option:text,text]
%TC:macro \citet [option:text,text]
%TC:envir table 0 1
%TC:envir table* 0 1
%TC:envir tabular [ignore] word
%TC:envir displaymath 0 word
%TC:envir math 0 word
%TC:envir comment 0 0
%%
%% The first command in your LaTeX source must be the \documentclass
%% command.
%%
%% For submission and review of your manuscript please change the
%% command to \documentclass[manuscript, screen, review]{acmart}.
%%
%% When submitting camera ready or to TAPS, please change the command
%% to \documentclass[sigconf]{acmart} or whichever template is required
%% for your publication.
%%
%%
\documentclass[sigconf,natbib=false]{acmart}
\usepackage[style=acmnumeric,backend=biber,sortcites=true]{biblatex}
\addbibresource{bibfile.bib}
\usepackage{lmodern}
\newcommand{\sect}[1]{\section{#1}}
\newcommand{\subsect}[1]{\subsection{#1}}
\newcommand{\subsubsectc}[1]{\subsubsection{#1}}
\newcommand{\subsubsubsect}[1]{\paragraph{#1}}
\newcommand{\subsubsubsubsect}[1]{\subparagraph{#1}}
\usepackage{tikz}
\usetikzlibrary{math}
\usetikzlibrary{arrows.meta}
\usetikzlibrary{positioning}
\usepackage{ifthen}
\tikzset{>={Stealth[length=1.25mm]}}
\newcommand{\todocite}{{\color{red}CITE}}
\newcommand{\todo}[1]{{\color{red}TODO: #1}}
\usepackage[per-mode=symbol]{siunitx}


%% Rights management information.  This information is sent to you
%% when you complete the rights form.  These commands have SAMPLE
%% values in them; it is your responsibility as an author to replace
%% the commands and values with those provided to you when you
%% complete the rights form.
\setcopyright{rightsretained}
\copyrightyear{2025}

%%
%% end of the preamble, start of the body of the document source.
\begin{document}

%%
%% The "title" command has an optional parameter,
%% allowing the author to define a "short title" to be used in page headers.
\title{Implementing Distributed In-Memory Trees on FPGAs
}

\author{Clarity Shimoniak}
\email{clarity.shimoniak@email.ucr.edu}
\affiliation{%
	\institution{University of California, Riverside}
	\city{Riverside}
	\state{California}
	\country{USA}
}

%%
%% The abstract is a short summary of the work to be presented in the
%% article.
\begin{abstract}
% Abstract should be double-spaced and limited to 350 words or 2,450 characters.
Both FPGAs and RDMA have seen increasing adoption in datacenters as a means of achieving the parallelism and responsiveness needed in modern applications. We propose a novel database architecture built around a distributed FPGA cluster with RDMA as its interconnect in order to implement high-performance, in-memory key-value store based on the B-Link tree.
\end{abstract}


%%
%% This command processes the author and affiliation and title
%% information and builds the first part of the formatted document.
\maketitle

../common/content.tex

%%
%% The acknowledgments section is defined using the "acks" environment
%% (and NOT an unnumbered section). This ensures the proper
%% identification of the section in the article metadata, and the
%% consistent spelling of the heading.
\begin{acks}
I would like to express my deepest appreciation to my advisor Philip Brisk,
without whose guidance, feedback, support, and advocacy I would not be here. Not
all of the challenges I faced have been expected, but he has supported me
through all of them.

Additionally, I would like to express my gratitude to Prithviraj Yuvaraj, whose
support navigating Vitis HLS software toolchain and the OCT testbed have been
invaluable to my efforts in implementing the system designed here.

Finally, I am grateful to my friends in Kansas for telling me about Zotero,
which has been immensely useful in keeping track of all my references.

\end{acks}


\printbibliography


%%
%% If your work has an appendix, this is the place to put it.
../common/appendix.tex


\end{document}
\endinput
%%
%% End of file `sample-sigconf.tex'.
